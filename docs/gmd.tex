\documentclass[11pt]{report}
\usepackage{ifthen}
\newcommand{\gmd}{\texttt{gmd}}
\newcommand{\tag}[1]{\texttt{<#1>}}
\newcommand{\gat}[1]{\texttt{<\textbackslash{}#1>}}
\newcommand{\supporting}[1]{\textbf{Supporting Drivers: }\texttt{#1}}
\newcommand{\element}[4][0in]{%
  \indent\hspace{#1}\tag{#2 #3}\\%
  \indent\hspace{.2in}#4\\%
  \indent\hspace{#1}\gat{#2}}
\newcommand{\attr}[4]{%
  \ifthenelse{\equal{#3}{}}{%
    \ifthenelse{\equal{#4}{}}{%
      \texttt{#1=``#2''}}{%
      \texttt{#1=``#2\{#4\}''}}}{%
    \ifthenelse{\equal{#4}{}}{%
      \texttt{#1=``#2[#3]''}}{%
      \texttt{#1=``#2[#3]\{#4\}''}}}}


\title{\gmd{} User's Guide}
\author{Tim Fuller}

\begin{document}
\maketitle

% ----------------------------------------------------------------------------- %
\chapter{Introduction}
\gmd{} is a Generalized Model Driver.

% ----------------------------------------------------------------------------- %
\chapter{Building \gmd}
\gmd's code base is largely written in Python and requires no additional
compiling.  However, the exodusII third party library and material models
written in fortran must be built.

% ----------------------------------------------------------------------------- %
\section{Setting Up}
Set up and build the third party libraries.
\begin{verbatim}
% cd GMD/toolset
% python setup.py
\end{verbatim}

Add \texttt{GMD/toolset} to \texttt{PATH}.

% ----------------------------------------------------------------------------- %
\section{Building}
Build the material libraries
\begin{verbatim}
% buildmtls
\end{verbatim}

% ----------------------------------------------------------------------------- %
\chapter{Running}
Make sure \texttt{GMD/toolset} is on your \texttt{PATH}.
\begin{verbatim}
% gmd runid[.xml]
\end{verbatim}

The following files will be produced
\begin{verbatim}
% ls runid.*
runid.exo       runid.log       runid.xml
\end{verbatim}
%
\texttt{runid.exo} is the exodusII output database, \texttt{runid.log} the log
file, and \texttt{runid.xml} the input file.


% ----------------------------------------------------------------------------- %
\chapter{User Input}
User input is via xml control files. In general, tags use CamelCase and
attributes lower case.  Attributes are described in this document as
%
\begin{verbatim}
attr="type[default]{choices}"
\end{verbatim}
%
where \texttt{default} is the default value (if any) and \texttt{\{choices\}}
are valid choices (if any). Any attribute not having a default value are
required. Types are \texttt{str}, \texttt{int}, \texttt{real}, \texttt{list}.
Lists are given as space separated lists (i.e., ``1 2 3'').


% ----------------------------------------------------------------------------- %
\section{GMDSpec}
\begin{verbatim}
<GMDSpec>
\end{verbatim}
%
All input files must have as their root element \tag{GMDSpec}. Recognized
subelements of \tag{GMDSpec} are
%
\begin{itemize}
  \item \tag{Physics}
  \item \tag{Permutation}
  \item \tag{Optimization}
\end{itemize}

Additionally, the following elements are read from anywhere in the input file
%
\begin{itemize}
  \item \tag{Include}
  \item \tag{Function}
  \item \tag{TerminationTime}
\end{itemize}

% ----------------------------------------------------------------------------- %
\section{Preprocessing}
Preprocessing allows specifying variables in the input inside of comment tags
for use in other parts of the input. Syntax mirrors that of \texttt{aprepro}.

\subsection{Example}
Specify the \tag{Material} parameter \texttt{K} and \tag{Path} parameter
\texttt{estar} as variables
\begin{verbatim}
<GMDSpec>
  <!-- {K = 23e9}
       {estar = -.05}
  -->
  <Physics>
    <Material model="elastic">
      <K> {K} </K>
      <G> 54e9 </G>
    </Material>
    <Path type="prdef" estar="{estar}">
      ...
    </Path>
  </Physics>
</GMDSpec>
\end{verbatim}

% ----------------------------------------------------------------------------- %
\section{Include}
\begin{verbatim}
<Include href="str"/>
\end{verbatim}
%
Path to file to be included as if its contents were inplace in the input file

\subsection{Example}
\begin{verbatim}
<Include href="/path/to/some/file.ext"/>
\end{verbatim}

% ----------------------------------------------------------------------------- %
\section{Function}
\begin{verbatim}
<Function id="int"
          type="str{analytic expression, piecewise linear}"
          var="str[x]" href="str" cols="list[1 2]">
\end{verbatim}
%
Define functions to be used elsewhere in input. \texttt{id=0} and
\texttt{id=1} are reserved for the constant $0$ and $1$ functions,
respectively.

% ----------------------------------------------------------------------------- %
\subsection{Examples}
\paragraph{Analytic expression}
%
\begin{verbatim}
<Function id="2" type="analytic expression" var="t">
  sin(t)
</Function>
\end{verbatim}

\paragraph{Piecewise linear table}
%
\begin{verbatim}
<Function id="2" type="piecewise linear">
  1 2
  2 3
  3 5
</Function>
\end{verbatim}
%
Read a piecewise linear table from an external file using columns 1 and 3
%
\begin{verbatim}
<Function id="2" type="piecewise linear" href="./file.dat" cols="1 3"/>
\end{verbatim}

\begin{verbatim}
% cat file.dat
# Column1 Column2 Column3
1 1 4
2 3 7
.
.
.
100 4.2 1.43
\end{verbatim}

% ----------------------------------------------------------------------------- %
\section{TerminationTime}
\begin{verbatim}
<TerminationTime> float </TerminationTime>
\end{verbatim}
%
Termination time for simulation.  If not specified, termination time is taken
as final time in \tag{Path}.

\subsection{Example}
\begin{verbatim}
<TerminationTime> 1.e-6 </TerminationTime>
\end{verbatim}

% ----------------------------------------------------------------------------- %
\section{Physics}
\begin{verbatim}
<Physics driver="str[solid]{solid, eos}">
\end{verbatim}
%
Define the physics of the simulation. Recognized subelements of \tag{Physics}
are
%
\begin{itemize}
  \item \tag{Path}
  \item \tag{Surface}
  \item \tag{Material}
  \item \tag{Extract}
\end{itemize}

% ----------------------------------------------------------------------------- %
\subsection{Path}
\supporting{solid}
\begin{verbatim}
<Path type="str{prdef}"
      format="str[default]{default, table, fcnspec}"
      cols="list[1, ..., n]" cfmt="str" tfmt="time"
      nfac="int[1]" kappa="real[0]"
      tstar="real[1]" estar="real[1]" sstar="real[1]"
      amplitude="real[1]" ratfac="real[1]" href="str">
\end{verbatim}
%
Define deformation paths. The jth leg of \tag{Path} is sent to the driver in
form \texttt{[tf, n, cfmt, Cij]}, where \texttt{tf}, \texttt{n},
\texttt{cfmt}, and \texttt{Cij} are the termination time, number of steps,
control format, and control values.

A note on \texttt{cfmt} and \texttt{Cij}. \texttt{cfmt[i]} instructs the
driver as to the type of deformation represented by \texttt{Cij[i]}. Supported
\texttt{cfmt} are described in Table \ref{tab:cfmt}. For example, the
following \texttt{cfmt} instructs the driver that the components of
\texttt{Cij} represent [stress, strain, stress rate, strain rate, strain,
strain], respectively: \texttt{cfmt = 423122}.  Mixed modes are allowed only
for components of strain rate, strain, stress rate, and stress.  Electric
field components can be included with any deformation type.

The components \texttt{Cij} take the following order

\textbf{Vectors:} [X, Y, Z]

\textbf{Symmetric tensors:} [XX, YY, ZZ, XY, YZ, XZ]

\textbf{Tensors:} [XX, XY, XZ, YX, YY, YZ ZX, ZY, ZZ]

If \texttt{len(Cij) $\neq$ 6} (or 9 for deformation gradient), the missing
components are assumed to be zero strain.

\begin{table}[h!]
  \centering
  \begin{tabular}[h]{cl}
    \hline
    \hline
    \texttt{cfmt} & Deformation type \\
    \hline
    1 & Strain rate \\
    2 & Strain \\
    3 & Stress rate \\
    4 & Stress \\
    5 & Deformation gradient \\
    6 & Electric field
  \end{tabular}
  \caption{Supported deformation types and \texttt{cfmt} code}
  \label{tab:cfmt}
\end{table}


% ----------------------------------------------------------------------------- %
\subsubsection{Examples}
The following examples will help clarify the \tag{Path} input syntax

\paragraph{format: default} Uniaxial strain, all six components of strain prescribed
\begin{verbatim}
<Path type="prdef" kappa="0" tstar="1" estar="-.5" amplitude="1" ratfac="1">
  <!-- termination time, number of steps, cfmt, Cij -->
  0   0 222222 0 0 0 0 0 0
  1 100 222222 1 0 0 0 0 0
  2 100 222222 2 0 0 0 0 0
  3 100 222222 1 0 0 0 0 0
  4 100 222222 0 0 0 0 0 0
</Path>
\end{verbatim}

\paragraph{format: default} Uniaxial strain, stress controlled
\begin{verbatim}
<Path type="prdef" nfac="100">
  0 0 444 0 0 0
  1 1 444 -7490645504 -3739707392 -3739707392
  2 1 444 -14981291008 -7479414784 -7479414784
  3 1 444 -7490645504 -3739707392 -3739707392
  4 1 444 0 0 0
</Path>
\end{verbatim}

\paragraph{format: default} Uniaxial stress, mixed mode
\begin{verbatim}
<Path type="prdef" nfac="100">
  0 0 222 0 0 0
  1 1 244 {epsmax} 0 0
  4 1 244 0 0 0
</Path>
\end{verbatim}

\paragraph{format: table} Read legs from table. Control type is uniform for all
legs. Specify control type as \texttt{cfmt} attribute of \tag{Path}.
Optionally, specify the time format as \texttt{tfmt} and number of steps for
each leg as \texttt{nfac}.
\begin{verbatim}
<Path type="prdef" format="table" cols="1:4" cfmt="222" tfmt="time">
  0 0 0 0
  1 1 0 0
    ...
  n 2 0 0
</Path>
\end{verbatim}

\paragraph{format: table} Read the table from a file, first by the
\tag{Include} element and then the \texttt{href} attribute.
\begin{verbatim}
<Path type="prdef" format="table" cols="1 3:8" cfmt="222222" tfmt="time">
  <include href="exmpls.tbl"/>
</Path>
\end{verbatim}

\begin{verbatim}
<Path type="prdef" format="table" cols="1 3:8" cfmt="222222" tfmt="time"
      href="exmpls.tbl"/>
\end{verbatim}

\paragraph{format: fcnspec} Create legs from functions. Functions are specified
as \texttt{function id[:scale]}.  Syntax is otherwise similar to table format.
Only a single leg can be specified.
\begin{verbatim}
<Path type="prdef" kappa="0" tstar="1" amplitude="1" format="fcnspec"
      cfmt="222" nfac="200">
  {2 * pi} 2:1.e-1 1:0 1:0
</Path>
\end{verbatim}

% ----------------------------------------------------------------------------- %
\subsection{Surface}
\supporting{eos}
\begin{verbatim}
<Surface format="str[default]{default, table}"
         cols="list[1, ..., n]" cfmt="str"
         nfac="int[1]" tstar="real[1]" rstar="real[1]"
         amplitude="real[1]" href="str">
\end{verbatim}
%
Define equaiton of state surface boundaries.  Input is similar to the
\tag{Path} specification, but leg termination time is not specified.  Control
parameters also differ, as shown in Table \ref{tab:cfmt-1}.

\begin{table}[h!]
  \centering
  \begin{tabular}[h]{cl}
    \hline
    \hline
    \texttt{cfmt} & Variable type \\
    \hline
    1 & Density \\
    2 & Temperature
  \end{tabular}
  \caption{Supported surface variable types and \texttt{cfmt} code}
  \label{tab:cfmt-1}
\end{table}

% ----------------------------------------------------------------------------- %
\subsubsection{Examples}
The following examples demonstrate the \tag{Surface} input.

\paragraph{format: default}
\begin{verbatim}
<Surface>
  <!-- nsteps, control, Cij -->
  <!-- control goes as 1 -> density
                       2 -> temperature -->
    0 12 1 100
  100 12 5 300
</Surface>
\end{verbatim}

\paragraph{format: table}
\begin{verbatim}
<Surface format="table" cfmt="12" nfac="100">
  <!-- Cij -->
  1 100
  5 300
</Surface>
\end{verbatim}

% ----------------------------------------------------------------------------- %
\subsection{Material}
\begin{verbatim}
<Material model="str">
\end{verbatim}
%
Specify the material model and parameters. Subelements of \tag{Material} are
%
\begin{itemize}
  \item \tag{Matlabel}
  \item \tag{Key}
\end{itemize}
%
Where \tag{Key} is a valid material parameter name.

\subsubsection{Matlabel}
\begin{verbatim}
<Matlabel href="str[F_MTL_PARAM_DB]">
\end{verbatim}
%
Insert model parameters from a database file.  The default file
\verb|F_MTL_PARAM_DB| is in \verb|/path/to/gmd/materials/material_properties.db|.

\subsubsection{Key}
\begin{verbatim}
<Key> float </Key>
\end{verbatim}

\subsubsection{Examples}
\begin{verbatim}
<Material model="elastic">
  <G>  54E+09 </G>
  <K> 124E+09 </K>
</Material>
\end{verbatim}

\begin{verbatim}
<Material model="elastic">
  <Matlabel href="./materials.xml"> aluminum </Matlabel>
  <K> 124E+09 </K>
</Material>
\end{verbatim}

% ----------------------------------------------------------------------------- %
\subsection{Extract}
\begin{verbatim}
<Extract format="str[ascii]{ascii, mathematica}" step="int[1]" ffmt="str[.18f]">
\end{verbatim}
%
Extract variables and paths from exodus output and (optionally) write to
different formats. Recognized subelements of \tag{Extract} are
%
\begin{itemize}
  \item \tag{Path}
  \item \tag{Variables}
\end{itemize}

\subsubsection{Variables}
\begin{verbatim}
<Variables> VAR_1, ..., VAR_N </Variables>
\end{verbatim}
%
Variables to extract from the exodus output database. Variables are specified
children of the \tag{Variables} element. All components of vector and tensor
variables will be extracted if only the basename is specified. Time is always
extracted as the first entry of the output file.  Extracted variables are in
\texttt{runid.out} or \texttt{runid.math} depending if the format is ascii or
mathematica.

\subsubsection{Path}
\supporting{eos}
\begin{verbatim}
<Path type="str{isotherm, hugoniot}" increments="int[100]"
      density_range="list" initial_temperature="real">
\end{verbatim}
%
Extract a specified path from the equation of state surface through the
specified density range starting at the initial temperature.

\subsubsection{Examples}
Extract all components of stress and strain
%
\begin{verbatim}
<Extract format="ascii">
  <variables>
    STRESS STRAIN
  </variables>
</Extract>
\end{verbatim}

Extract only the XX, YY, and ZZ components of stress
%
\begin{verbatim}
<Extract format="ascii">
  <variables>
    STRESS_XX STRESS_YY STRESS_ZZ
  </variables>
</Extract>
\end{verbatim}

Extract all variables
\begin{verbatim}
<Extract format="ascii">
  <variables>
    ALL
  </variables>
</Extract>
\end{verbatim}

Extract Hugoniot and Isotherm paths
\begin{verbatim}
<Extract>
  <Path type="isotherm" increments="200"
        density_range="1 3" initial_temperature="225"/>
  <Path type="hugoniot" increments="100"
        density_range="1 3" initial_temperature="100"/>
</Extract>
\end{verbatim}

% ----------------------------------------------------------------------------- %
\section{Permutation}
\begin{verbatim}
<Permutation method="str[zip]{zip, combine}" seed="real[12]">
\end{verbatim}
%
Permutate model input parameters to investigate sensitivities. Recognized
subelements of \tag{Permutation}

\begin{itemize}
  \item \tag{Permutate}
  \item \tag{ResponseFunction}
\end{itemize}

% ----------------------------------------------------------------------------- %
\subsection{Permutate}
\begin{verbatim}
<Permutate var="str"
           values="str{range, list, weibull, uniform, normal, percentage}"
\end{verbatim}
%
Specify the paramaters to permutate. Variable names should occur elsewhere in
the input file in preprocessing braces.

\subsection{Example}
Permutate the \texttt{K} and \texttt{G} parameters
%
\begin{verbatim}
<Permutation method="zip" seed="12">
  <Permutate var="K" values="weibull(125.e9, 14, 3)"/>
  <Permutate var="G" values="percentage(45.e9, 10, 3)"/>
</Permutation>
\end{verbatim}

In the \tag{Material} element, the \texttt{K} and \texttt{G} parameters are
specified as
%
\begin{verbatim}
<Material model="elastic">
  <K> {K} </K>
  <G> {G} </G>
</Material>
\end{verbatim}

% ----------------------------------------------------------------------------- %
\subsection{ResponseFunction}
\begin{verbatim}
<ResponseFunction href="str" descriptor="str[]"/>
\end{verbatim}
Name of response function that returns the response from permutation or
optimization jobs.  \texttt{href} must either be the path to a file containing
the response function, or the name of a builtin \gmd{} response function.

Built in response functions are
%
\begin{itemize}
  \item \texttt{gmd.max}
  \item \texttt{gmd.min}
  \item \texttt{gmd.mean}
  \item \texttt{gmd.ave}
\end{itemize}
%
Built in response functions operate only on variabes in the simulation output file.

If \texttt{href} is a user defined script, the script is called from the
command line as
\begin{verbatim}
% ./scriptname simulation_output.exo [auxiliary_file_1 [... auxiliary_file_n]]
\end{verbatim}

\subsubsection{Examples}
\begin{verbatim}
<ResponseFunction href="./scriptname" descriptor="PRES"/>
\end{verbatim}


\begin{verbatim}
<ResponseFunction href="gmd.max(PRESSURE)" descriptor="PRES"/>
\end{verbatim}

% ----------------------------------------------------------------------------- %
\section{Optimization}
\begin{verbatim}
<Optimization method="str[simplex]{simplex, powell, cobyla}"
              maxiter="int[25]" tolerance="real[1e-6]">
\end{verbatim}
%
Optimize specified parameters against user specified objective function.
Recognized subelements of \tag{Optimization}

\begin{itemize}
  \item \tag{Optimize}
  \item \tag{AuxiliaryFile}
  \item \tag{ResponseFunction}
\end{itemize}

% ----------------------------------------------------------------------------- %
\subsection{Optimize}
\begin{verbatim}
<Optimize var="str" initial_value="real" bounds="list[]"/>
\end{verbatim}
%
Specify the variable to be optimized, giving initial value and, optionally,
bounds.  Only the \texttt{cobyla} method accepts bounds.  Variable names should
occur elsewhere in the input file in preprocessing braces.

% ----------------------------------------------------------------------------- %
\subsection{ResponseFunction}
Same as for \tag{Permutation}.  The value returned from the response function
is interpreted as the error to be minimized.

% ----------------------------------------------------------------------------- %
\subsection{AuxiliaryFile}
\begin{verbatim}
<AuxiliaryFile href="str"/>
\end{verbatim}
Path to any auxiliary file needed by the optimization objective function.

\subsection{Example}
Optimize the \texttt{K} and \texttt{G} parameters
\begin{verbatim}
<Optimization method="simplex" maxiter="25" tolerance="1e-4" disp="0">
  <ResponseFunction href="opt-sig-v-time" descriptor="SIG_V_TIME"/>
  <AuxiliaryFile href="opt-baseline.dat"/>
  <Optimize var="opt_k" initial_value="129.e9"/>
  <Optimize var="opt_g" initial_value="54.e9"/>
</Optimization>
\end{verbatim}

In the \tag{Material} element, the \texttt{K} and \texttt{G} parameters are
specified as
%
\begin{verbatim}
<Material model="elastic">
  <K> {opt_k} </K>
  <G> {opt_g} </G>
</Material>
\end{verbatim}

\end{document}

%%% Local Variables:
%%% mode: latex
%%% TeX-master: t
%%% End:
